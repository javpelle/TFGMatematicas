\usepackage[utf8]{inputenc}
\usepackage[spanish]{babel}
\usepackage{graphicx}

\setlength{\parskip}{1em}
\setlength{\parindent}{1em}

\xdefinecolor{rojito}{rgb}{1,0.3,0.3}
\xdefinecolor{oliva}{cmyk}{0.64,0,0.95,0.4}
\xdefinecolor{minaranja}{rgb}{0.94,0.48,0.2}

\usetheme{Madrid}
\usecolortheme[named=rojito]{structure}

\usepackage{mathtools}
\usepackage{amsmath}
%------------------------
%------- TEOREMAS -------
%------------------------

\newtheorem{thm}{Teorema} % Estilo de texto cursivo

\newtheorem{postulate}{Postulado} % Numeración general sin subíndices

\title[Computación Cuántica con Qiskit]{Introducción a la Computación Cuántica con Qiskit}
\author{Javier Pellejero Ortega}
\institute[UCM]{Universidad Complutense de Madrid\\ Facultad de Matemáticas}

% Conjuntos y constantes matemáticas
\newcommand{\R}{\mathbb{R}} % Reales
\newcommand{\C}{\mathbb{C}} % Complejos
\newcommand{\K}{\mathbb{K}} % Cuerpo K (reales o complejos)
\newcommand{\Sp}{\mathbb{S}} % Esfera
\newcommand{\e}{\mathrm{e}} % número e
\newcommand{\N}{\mathbb{N}} % Naturales
\newcommand{\Q}{\mathbb{Q}} % Racionales
\newcommand{\B}{\mathcal{B}} % Base

\newcommand{\qsh}{\textsf{Q}\texttt{\#}} % Q#
\newcommand{\csh}{\textsf{C}\texttt{\#}} % C#

\newcommand{\orden}[1]{\mathcal{O}\left(#1\right)}

\newcommand{\oversim}[1]{\overset{_\sim}{#1}} % Para poner ~ sobre algo

% Para poner datos encima y/o debajo de implica
\newcommand{\ximplies}[2]{\underset{#2}{\overset{#1}\implies}}
\newcommand{\xiff}[2]{\underset{#2}{\overset{#1}\iff}}
\newcommand{\ximpliedby}[2]{\underset{#2}{\overset{#1}\impliedby}}

% Producto escalar y norma
\newcommand{\dotproduct}[2]{\langle#1,#2\rangle}
\newcommand{\norm}[1]{\left|\left|#1\right|\right|}

% Notación de Dirac -- SUSTITUIR POR PAQUETE DE LUIS
\newcommand{\ket}[1]{\left|#1\right\rangle}
\newcommand{\bra}[1]{\left\langle#1\right|}
\newcommand{\braket}[2]{\left\langle#1|#2\right\rangle}

% Vectores
\newcommand{\twovector}[2]{\begin{pmatrix} #1 \\ #2 \end{pmatrix}} % Vector de dim 2

% Funciones e info debajo de funciones
\newcommand{\function}[3]{#1\colon #2\longrightarrow #3}
\newcommand{\xfunction}[4]{\underset{#4}{{#1\colon #2\longrightarrow #3}}}

% Funcion que transforma estados |0> y |1>
\newcommand{\gatetwo}[3]{#1\colon \begin{matrix}\ket0&\longrightarrow& #2\\ \ket1&\longrightarrow& #3\end{matrix}}

% Funcion que transforma estados |00>, |01>, |10> y |11>
\newcommand{\gatefour}[5]{#1\colon \begin{matrix} \ket{00}&\longrightarrow& #2\\ \ket{01}&\longrightarrow& #3\\ \ket{10}&\longrightarrow& #4\\ \ket{11}&\longrightarrow& #5 \end{matrix}}