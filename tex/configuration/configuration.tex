\documentclass[11pt, twoside]{book}
%------------------------
%------- PAQUETES -------
%------------------------

% Establecemos el tamaño de página y márgenes
\usepackage[a4paper, total={6in, 8in}]{geometry}

% ----- Establecemos la distancia entre párrafos ------

% El paquete parskip no sólo añade espacio entre párrafos sino que elimina
% también la identación de la primera línea de % cada uno. Como no queremos
% esto último guardamos la conf. de la identación antes de cargar el paquete.
\edef\restoreparindent{\parindent=\the\parindent\relax}

% Cargamos el paquete
\usepackage{parskip}

% Restauramos la identación
\restoreparindent

%\setlength{\parskip}{1em} % Sustituido por las instrucciones anteriores
% -----------------------------------------------------

\usepackage[utf8]{inputenc}
\usepackage[spanish]{babel}
\usepackage[T1]{fontenc}

% Bibliografía (No funciona Biber con Miktex (?))
\usepackage[backend=bibtex,style=authoryear,citestyle=authoryear]{biblatex}
\usepackage{csquotes}

% Visualización de marcadores en el lector PDF
\usepackage{bookmark}

% \overset, numeración de teoremas y otras
\usepackage{amsmath}

% Para cambiar numeración de "enumerate"
\usepackage{enumitem}

% Para \mathbb etc.
\usepackage{amssymb}

% Para inserción de imágenes
\usepackage{graphicx}

% Para definición de teoremas
\usepackage{amsthm}

% Para circuitos cuanticos. Requiere tener en el directorio el archivo qcircuit.sty
\usepackage{qcircuit}


% -------------------- Para Código --------------------
\usepackage{listings}
\usepackage[table,xcdraw]{xcolor} % tambíen incluye colores para tablas.

\definecolor{codegreen}{rgb}{0,0.6,0}
\definecolor{codegray}{rgb}{0.5,0.5,0.5}
\definecolor{codepurple}{rgb}{0.58,0,0.82}
\definecolor{backcolour}{rgb}{0.95,0.95,0.92}

\lstdefinestyle{mystyle}{
    backgroundcolor=\color{backcolour},   
    commentstyle=\color{codegreen},
    keywordstyle=\color{magenta},
    numberstyle=\tiny\color{codegray},
    stringstyle=\color{codepurple},
    basicstyle=\ttfamily\footnotesize,
    breakatwhitespace=false,         
    breaklines=true,                 
    captionpos=b,                    
    keepspaces=true,                 
    numbers=left,                    
    numbersep=5pt,                  
    showspaces=false,                
    showstringspaces=false,
    showtabs=false,                  
    tabsize=2
}

\lstset{style=mystyle}
% -----------------------------------------------------

% Para eliminar encabezado en páginas pares en blanco antes de nuevo capítulo
% -----------------------------------------------------
\makeatletter
\def\cleardoublepage{\clearpage\if@twoside \ifodd\c@page\else
    \hbox{}
    \thispagestyle{plain}
    \newpage
    \if@twocolumn\hbox{}\newpage\fi\fi\fi}
\makeatother \clearpage{\pagestyle{plain}\cleardoublepage}
% -----------------------------------------------------

%------------------------
%------- TEOREMAS -------
%------------------------

\newtheorem{thm}{Teorema}[chapter] % Estilo de texto cursivo

\theoremstyle{definition} % Cambio del estilo de los teoremas a normal
\newtheorem{proposition}[thm]{Proposición}
\newtheorem{definition}[thm]{Definición}
\newtheorem{corolario}[thm]{Corolario}
\newtheorem{example}[thm]{Ejemplo}
\newtheorem{observation}[thm]{Observación}
\newtheorem{postulate}{Postulado} % Numeración general sin subíndices
\newtheorem*{lema}{Lema}
\newtheorem*{nota}{Nota}

%------------------------
%------- COMANDOS -------
%------------------------

% Conjuntos y constantes matemáticas
\newcommand{\R}{\mathbb{R}} % Reales
\newcommand{\C}{\mathbb{C}} % Complejos
\newcommand{\K}{\mathbb{K}} % Cuerpo K (reales o complejos)
\newcommand{\Sp}{\mathbb{S}} % Esfera
\newcommand{\e}{\mathrm{e}} % número e
\newcommand{\N}{\mathbb{N}} % Naturales
\newcommand{\Q}{\mathbb{Q}} % Racionales
\newcommand{\B}{\mathcal{B}} % Base

\newcommand{\orden}[1]{\mathcal{O}\left(#1\right)}

\newcommand{\modulo}{\textrm{ mod }}

\newcommand{\oversim}[1]{\overset{_\sim}{#1}}
\newcommand{\sucesion}[2]{\left\{#1_{#2}\right\}^\infty_{#2 = 1}}
\newcommand{\sucesionelement}[2]{\left\{#1\right\}^\infty_{#2 = 1}}

% Para poner datos encima y/o debajo de implica
\newcommand{\ximplies}[2]{\underset{#2}{\overset{#1}\implies}}
\newcommand{\xiff}[2]{\underset{#2}{\overset{#1}\iff}}
\newcommand{\ximpliedby}[2]{\underset{#2}{\overset{#1}\impliedby}}

% Producto escalar y norma
\newcommand{\dotproduct}[2]{\langle#1,#2\rangle}
\newcommand{\norm}[1]{\left|\left|#1\right|\right|}

% Notación de Dirac
\newcommand{\ket}[1]{\left|#1\right\rangle}
\newcommand{\bra}[1]{\left\langle#1\right|}
\newcommand{\braket}[2]{\left\langle#1|#2\right\rangle}

% Vectores
\newcommand{\twovector}[2]{\begin{pmatrix} #1 \\ #2 \end{pmatrix}} % Vector de dim 2

% Funciones y info debajo de funciones
\newcommand{\function}[3]{#1\colon #2\longrightarrow #3}
\newcommand{\xfunction}[4]{\underset{#4}{{#1\colon #2\longrightarrow #3}}}

% Funcion que transforma estados |0> y |1>
\newcommand{\gatetwo}[3]{#1\colon \begin{matrix}\ket0&\longrightarrow& #2\\ \ket1&\longrightarrow& #3\end{matrix}}

% Funcion que transforma estados |00>, |01>, |10> y |11>
\newcommand{\gatefour}[5]{#1\colon \begin{matrix} \ket{00}&\longrightarrow& #2\\ \ket{01}&\longrightarrow& #3\\ \ket{10}&\longrightarrow& #4\\ \ket{11}&\longrightarrow& #5 \end{matrix}}

% Introducir información de igualdad debajo
\newcommand{\equals}[2]{\underset{#2}{\underset{\shortparallel}{#1}}}

% Comandos para palabras dentro de ecuaciones $ $ o \[ \]
\newcommand{\y}{\mathrm{\ y\ }}
\newcommand{\cte}{\mathrm{cte}}
\newcommand{\rg}{\mathrm{rg}}
\newcommand{\id}{\mathrm{id}}

\newcommand\blankpage{%
    \null
    \thispagestyle{empty}%
    \addtocounter{page}{-1}%
    \newpage}


\title{Introducción a la Computación Cuántica con Qiskit}
\author{Javier Pellejero Ortega}