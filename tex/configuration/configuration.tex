\documentclass[11pt, oneside]{book}
%------------------------
%------- PAQUETES -------
%------------------------

\usepackage[a4paper, total={6in, 8in}]{geometry}
\setlength{\parskip}{1em}

\usepackage[utf8]{inputenc}
\usepackage[spanish]{babel}
\usepackage[T1]{fontenc}

% Citas a pie de página de bibliografía
\usepackage[backend=bibtex, style=alphabetic, citestyle=verbose]{biblatex}
\usepackage{csquotes}

% \overset y otras
\usepackage{amsmath}

% Para \mathbb etc.
\usepackage{amssymb}

% Para inserción de imágenes
\usepackage{graphicx}

% Para definición de teoremas
\usepackage{amsthm}

%------------------------
%------- TEOREMAS -------
%------------------------

\newtheorem{teor}{Teorema} % Estilo de texto cursivo

\theoremstyle{definition} % Cambio del estilo de los teoremas a normal
\newtheorem{proposicion}{\textbf{Proposici\'on}}
\newtheorem{defi}{\textbf{Definición}}
\newtheorem{corolario}{Corolario}
\newtheorem{figura}{Figura}
\newtheorem*{lema}{Lema}
\newtheorem*{ejem}{\underline{Ejemplo}}
\newtheorem*{nota}{Nota}
\newtheorem*{observacion}{Observación}

%------------------------
%------- COMANDOS -------
%------------------------

% Conjuntos y constantes matemáticas
\newcommand{\R}{\mathbb{R}}
\newcommand{\e}{\mathrm{e}}
\newcommand{\N}{\mathbb{N}}
\newcommand{\Q}{\mathbb{Q}}
\newcommand{\C}{\mathbb{C}}
\newcommand{\orden}[1]{\mathcal{O}\left(#1\right)}

\newcommand{\oversim}[1]{\overset{_\sim}{#1}}
\newcommand{\sucesion}[2]{\left\{#1_{#2}\right\}^\infty_{#2 = 1}}
\newcommand{\sucesionelement}[2]{\left\{#1\right\}^\infty_{#2 = 1}}

% Para poner datos encima y/o debajo de implica
\newcommand{\ximplies}[2]{\underset{#2}{\overset{#1}\implies}}
\newcommand{\xiff}[2]{\underset{#2}{\overset{#1}\iff}}
\newcommand{\ximpliedby}[2]{\underset{#2}{\overset{#1}\impliedby}}

% Producto escalar y norma
\newcommand{\dotproduct}[2]{<#1,#2>}
\newcommand{\norm}[1]{\left|\left|#1\right|\right|}

% Funciones y info debajo de funciones
\newcommand{\function}[3]{#1\colon #2\longrightarrow #3}
\newcommand{\xfunction}[4]{\underset{#4}{{#1\colon #2\longrightarrow #3}}}

% Introducir información de igualdad debajo
\newcommand{\equals}[2]{\underset{#2}{\underset{\shortparallel}{#1}}}

% Comandos para palabras dentro de ecuaciones $ $ o \[ \]
\newcommand{\y}{\mathrm{\ y\ }}
\newcommand{\cte}{\mathrm{cte}}
\newcommand{\rg}{\mathrm{rg}}
\newcommand{\id}{\mathrm{id}}

\newcommand\blankpage{%
    \null
    \thispagestyle{empty}%
    \addtocounter{page}{-1}%
    \newpage}


\title{Introducción a la Computación Cuántica con Qiskit}
\author{Javier Pellejero Ortega}