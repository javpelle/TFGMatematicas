\chapter{Introducción a la computación cuántica}

Hemos desarrollado una serie de herramientas matemáticas y ahora toca desarrollar por qué lo hemos hecho. Esto no es tarea fácil pues se mete de lleno en el marco teórico de la mecánica cuántica. Este marco se sustenta principalmente en 6 postulados, mientras que \textit{Michael A. Nielsen} y \textit{Isaac L. Chuang} establecen 4 a partir de los anteriores [\cite{nielsen2001quantum}].

Estos cuatro postulados serán los cimientos de nuestra teoría de la computación cuántica. Antes de exponerlos, vamos a comentar un tipo de notación muy empleada en el mundo de la física cuántica y que usaremos constantemente a partir de ahora.

\section{Notación de Dirac}

La \textbf{notación de Dirac} o \textbf{notación \textit{bra-ket}} sirve para denotar un estado cuántico que, como veremos a continuación, no es más que un vector con coeficientes complejos. Veamos algunas de sus notaciones y propiedades.
\begin{itemize}
\item $\ket{\psi}$ denota un estado cuántico y, matricialmente, no es más que un vector columna.
\item $\bra{\psi}$ denota el conjugado de $\ket{\psi}$ traspuesto, es decir un vector fila conjugado.
\item $\bra{\psi}\ket{\psi'}$ escrito también como $\braket{\psi}{\psi'}$ es por tanto el producto interno $\dotproduct{\psi'}{\psi}$.

\item $\ket{\psi}\bra{\psi'}$ es el denominado \textbf{producto externo}. Se corresponde con una matriz y por tanto con una aplicación.

\item $\ket{\psi}\otimes\ket{\psi'}$ escrito también como $\ket{\psi}\ket{\psi'}$ o $\ket{\psi\psi'}$ denota el producto tensorial de $\ket{\psi}$ y $\ket{\psi'}$.
\end{itemize}

\section{Postulados de Michael A. Nielsen y Isaac L. Chuang}

Vamos a proceder a enunciar los postulados de los que hablábamos al principio de esta sección.

\begin{postulate} Un sistema físico aislado es un espacio vectorial con coeficientes en los complejos dotado de producto interno (como vimos en el ejemplo \ref{ex:ex32} esto supone que se trata de un Espacio de \textit{Hilbert}). El sistema es completamente descrito por un estado que no es más que un vector unitario de dicho sistema.
\end{postulate}

El postulado no dice nada sobre la dimensión del sistema, definiremos en la próxima sección como será nuestro sistema más elemental.

\begin{postulate} La evolución de un sistema cuántico cerrado en dos instantes de tiempo viene dada por una transformación lineal unitaria. Es decir, el estado del sistema $\ket{\psi}$ en el instante $t_1$ se transforma en $\ket{\psi'}$ en $t_2$ por la aplicación lineal unitaria $U$ que depende únicamente de $t_1$ y $t_2$:
\begin{equation}
\ket{\psi'}=U\ket{\psi}
\end{equation}
\end{postulate}

\begin{postulate} Las medidas de un sistema cuántico son descritas por un conjunto $\{M_m \}$ de \textbf{aplicaciones de medida}. Cada índice $m$ se identifica con un posible valor para dicha medición que ocurre, para un estado $\psi$, con una probabilidad de 
\begin{equation}
p(m)=\bra{\psi}M_m^\dag M_m\ket{\psi}
\end{equation}

tras la medición, el estado del sistema se convierte en
\begin{equation}
\dfrac{M_m\ket{\psi}}{\sqrt{\bra{\psi}M_m^\dag M_m\ket{\psi}}}.
\end{equation}

Por otro lado, las aplicaciones de medida deben satisfacer la \textbf{ecuación de completitud} que enunciamos como:
\begin{equation}
\sum_m M_m^\dag M_m = I
\end{equation}

Por tanto, como $\psi$ tiene norma 1, se verifica que la suma de las probabilidades de cada valor de medición posible es 1.
\begin{equation}
\sum_m p(m)=\sum_m \bra{\psi}M_m^\dag M_m\ket{\psi} = 1.
\end{equation}
\end{postulate}

Lo primero que debemos resaltar de este postulado es que hay una transformación del estado cuántico del sistema que no viene dada por una aplicación unitaria. Sin embargo, estamos contemplando que el sistema, a diferencia de lo que ocurre en el postulado 2,  no es cerrado; Esto indica que la medición es una interacción externa.

Veremos en la siguiente sección un ejemplo de medida de estados ortonormales sobre un qubit. Es importante recalcar que no podremos definir mediciones distinguibles si los valores \textbf{no son ortogonales} [\cite[p.~87]{nielsen2001quantum}]. 

\begin{postulate} El sistema resultante de la composición de dos o más sistemas físicos es el producto tensorial de los mismos. Si denotamos a estos sistemas por $1,2,...,n$ representados por los vectores unitarios $\ket{\psi_i}$ para todo $i=1,2,...,n$, entonces el estado de sistema final total es $\ket{\psi_1}\otimes\ket{\psi_2}\otimes\hdots\otimes\ket{\psi_n}$.

Estos postulados justifican el desarrollo de los dos capítulos anteriores que no hacían más que exponer los conceptos matemáticos usados por estos axiomas y que consolidan toda la teoría de computación cuántica que explicaremos a partir de ahora.
\end{postulate}

\section{Qubits}

Un \textbf{qubit}  es el elemento de información mínima en la teoría de computación cuántica. No entraremos en cómo se puede representar dicho elemento físicamente en el mundo real.

Al igual que un bit clásico se corresponde con un estado que típicamente puede ser 0 o 1, lo mismo ocurre con un qubit. Denotamos a dichos posibles estados por $\ket{0}$ y $\ket{1}$ y se corresponderán respectivamente con los elementos de una base ortonormal de $\C^2$. De ser necesario darles forma vectorial, los identificaremos normalmente con los vectores $\twovector{1}{0}$ y $\twovector{0}{1}$.

Además de estos posibles estados, y es aquí donde reside la diferencia con respecto al bit clásico, un qubit puede estar en una combinación lineal de los mismos que denominamos \textbf{superposición} y denotamos por:
\begin{equation}\ket{\psi}=\alpha\ket{0}+\beta\ket{1}
\end{equation}
donde $\alpha$ y $\beta$ son números complejos que verifican $|\alpha|^2+|\beta|^2=1$. Podemos decir así que el estado de un qubit es un vector unitario de $\C^2$, o equivalentemente, un elemento de la 3-esfera, subconjunto de $\C^2$ que definimos como:
\begin{equation}\Sp^3=\left\{ (z_1,z_2)\in\C^2:|z_1|^2+|z_2|^2=1 \right\}
\end{equation}

Como vemos, hasta ahora nos hemos ceñido a definir un sistema físico cuántico que verifica las directrices del primer postulado. Veamos qué pasa cuando tratamos de medir un qubit y cómo se ajusta este proceso al postulado 3.

Un bit clásico puede ser medido para saber su estado en cualquier momento. Dicha medición otorgará, por tanto, uno de los dos posibles estados. La medición en un qubit con estado $\ket{\psi}=\alpha\ket{0}+\beta\ket{1}$ no nos posibilita conocer los valores de $\alpha$ y $\beta$. Vamos a definir las aplicaciones de medida correspondientes a los estados ortonormales $\ket{0}$ y $\ket{1}$.

\begin{equation}
M_{\ket{0}}=\ket{0}\bra{0}=\left(\begin{matrix}1&0\\0&0\end{matrix}\right)
\end{equation}
\begin{equation}
M_{\ket{1}}=\ket{1}\bra{1}=\left(\begin{matrix}0&0\\0&1\end{matrix}\right)
\end{equation}

Se verifica la ecuación de completitud ($M_{\ket{0}}^\dag M_{\ket{0}} + M_{\ket{1}}^\dag M_{\ket{1}}=I$) y la probabilidad de obtener $\ket{0}$ tras una medición es $p(\ket{0})=\bra{\psi}M_{\ket{0}}^\dag M_{\ket{0}}\ket{\psi}=\overline{\alpha}\alpha = |\alpha|^2$. Análogamente, $p(\ket{1})=|\beta|^2$. Se obtiene también la igualdad
\[p(\ket{0})+p(\ket{1})=|\alpha|^2+|\beta|^2=1\]

Por último, si la medición obtuvo $\ket{0}$, el estado del sistema tras la medida será:
\begin{equation}
\dfrac{M_{\ket{0}}\ket{\psi}}{\sqrt{\bra{\psi}M_{\ket{0}}^\dag M_{\ket{0}}\ket{\psi}}}=\dfrac{\alpha\ket{0}}{|\alpha|}.
\end{equation}

Mientras que en el caso de obtener $\ket{1}$, tendremos:
\begin{equation}
\dfrac{M_{\ket{1}}\ket{\psi}}{\sqrt{\bra{\psi}M_{\ket{1}}^\dag M_{\ket{1}}\ket{\psi}}}=\dfrac{\beta\ket{1}}{|\beta|}.
\end{equation}

Cabe señalar que los estados $e^{i\theta}\ket{\psi}$, con $\theta\in\R$ y $\ket{\psi}$ son \textbf{indistinguibles} [\cite[p.~93]{nielsen2001quantum}]. Basta ver que si $M_m$ es una aplicación de medida, entonces $\bra{\psi}e^{-i\theta}M_m^\dag M_m e^{i\theta}\ket{\psi}=\bra{\psi}M_m^\dag M_m\ket{\psi}$. Así, podemos sustituir en las anteriores ecuaciones $\dfrac{\alpha\ket{0}}{|\alpha|}$ y $\dfrac{\beta\ket{1}}{|\beta|}$ por $\ket{0}$ y $\ket{1}$ respectivamente.

Podemos definir otros conjuntos de medida para otras bases ortonormales como las que vimos en el ejemplo \ref{ex:ex34}. Denotamos a esos elementos como $\ket+=\dfrac{1}{\sqrt{2}}(\ket0 + \ket1)$ y $\ket-=\dfrac{1}{\sqrt{2}}(\ket0 - \ket1)$. Así, si aplicamos una medición a $\ket+$ con el conjunto de medidas $\{M_{\ket{0}},M_{\ket{1}}\}$ obtendremos $\ket0$ o $\ket1$ con probabilidad idéntica de $\frac{1}{2}$, pero si lo hacemos mediante $\{M_{\ket{+}},M_{\ket{-}}\}$ obtendremos $\ket+$ con un certeza absoluta.

Que el estado de un qubit no pueda ser observado como tal puede parecer algo poco intuitivo; que los posibles resultados obtenidos tras medir no difieran en número respecto a un bit clásico tampoco ayuda a creer que la computación cuántica ofrezca muchas ventajas sobre la clásica.

Sin embargo, veremos que los estados de un qubit podrán ser transformados, no sólo de manera individual, sino en conjunto con otros qubits formando un sistema, y que dichas transformaciones provocarán unas propiedades en los estados que usaremos a nuestro favor para obtener mediciones satisfactorias.

Antes de concluir esta sección vamos a resaltar de nuevo los estados cuánticos $\ket+$ y $\ket-$ por una propiedad que dejamos entrever antes, la de equiprobabilidad de obtener los valores $\ket{0}$ y $\ket{1}$ al medir.


\section{Sistemas de múltiples qubits}

Acabamos de ver qué es un qubit y hemos definido su estado como un vector unitario de $\C^2$ expresado como combinación lineal de la suma de dos elementos de una base ortonormal. En esta sección indagaremos sobre cómo es un sistema de múltiples qubits: cómo son sus estados, sus bases y sus posibles valores de medición.

Empecemos por el sistema. Un sistema de 2 qubits, en vista al cuarto postulado, será el producto tensorial de $\C^2\otimes\C^2$; Es decir, $\C^4$. Generalizando para el caso de $n$ qubits, el sistema resultante será un espacio equivalente a $\C^{2^n}$.

En cuanto a una base para el sistema, en el caso de dos qubits, una base estaría formada por el conjunto $\{\ket{0}\otimes\ket{0},\ket{0}\otimes\ket{1},\ket{1}\otimes\ket{0},\ket{1}\otimes\ket{1}\}$ o, equivalentemente, $\{\ket{00},\ket{01},\ket{10},\\ \ket{11}\}$. En el caso general, la base estaría compuesta por $\{\ket{0},\ket{1},...,\ket{2^n-1}\}$\footnote{Nótese que en este caso utilizamos, para simplificar, la notación decimal.}.

Sean ahora dos qubits con estados $\ket{\psi_1}=\alpha\ket{0}+\beta\ket{1}$ y $\ket{\psi_2}=\gamma\ket{0}+\delta\ket{1}$. Entonces, el estado del sistema formado por ambos qubits viene dado por:
\[\ket{\psi}=\ket{\psi_1}\otimes\ket{\psi_2}=\alpha\gamma\ket{00}+\alpha\delta\ket{01}+\beta\gamma\ket{10}+\beta\delta\ket{11}\]

Debemos comprobar que $\ket{\psi}$ sea unitario para que verifique el primer postulado. En efecto,
\[|\alpha\gamma|^2+|\alpha\delta|^2+|\beta\gamma|^2+|\beta\delta|^2=\alpha\gamma\overline{\alpha\gamma}+\alpha\delta\overline{\alpha\delta}+\beta\gamma\overline{\beta\gamma}+\beta\delta\overline{\beta\delta}=\alpha\overline{\alpha}(\gamma\overline{\gamma}+\delta\overline{\delta})+\beta\overline{\beta}(\gamma\overline{\gamma}+\delta\overline{\delta})=\]
\[=|\alpha|^2(|\gamma|^2+|\delta|^2)+|\beta|^2(|\gamma|^2+|\delta|^2)=|\alpha|^2+|\beta|^2=1\]

Definimos el caso general para $n$ qubits de la misma manera; Así, el estado resultante $\ket\psi=\alpha_0\ket{0\hdots0}+\hdots+\alpha_{2^n-1}\ket{1\hdots1}$ verifica $\sum_{i=0}^{2^n-1}|\alpha_i|^2=1$. Por tanto, si medimos dicho estado obtendremos el valor $\ket{i}$ con probabilidad $|\alpha_i|^2$ y, en ese caso, el estado colapsará tras la medición a $\ket{i}$.

Para acabar el capítulo señalaremos que algunos estados no pueden ser alcanzados por medio de la composición de dos sistemas por separado. Un ejemplo de esto es el estado $\ket{\psi}=\dfrac{1}{\sqrt{2}}(\ket{00}+\ket{11})$, no existen $\ket{\psi_1}=\alpha\ket{0}+\beta\ket{1}$ y $\ket{\psi_2}=\gamma\ket{0}+\delta\ket{0}$ tales que $\ket{\psi_1}\otimes\ket{\psi_2}=\ket{\psi}$. Si así fuera tendríamos
\[(\alpha\ket{0}+\beta\ket{1})\otimes(\gamma\ket{0}+\delta\ket{0})=\alpha\gamma\ket{00}+\alpha\delta\ket{01}+\beta\gamma\ket{10}+\beta\delta\ket{11}\]

y por tanto, para que $\ket{01}$ desaparezca de la ecuación, bien $\alpha =0$ o bien $\gamma=0$, pero esto anularía también $\ket{00}$ o $\ket{11}$. A este tipo de estados se les denomina como \textbf{estado entrelazado}. En caso contrario, si el estado de un sistema de más de un qubit puede ponerse como producto tensorial de estados correspondientes a sistemas más simples, lo denominamos \textbf{estado producto}.

% Estados entrelazados, estados no entrelazados