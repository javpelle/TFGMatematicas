\chapter*{Prefacio}
\addcontentsline{toc}{chapter}{Prefacio}

Este trabajo no pretende ser un curso de computación cuántica, tan sólo son unas notas que forman parte del proceso seguido por el autor para adentrase por primera vez en este mundo. Como alumno del Doble Grado de Matemáticas e Ingeniería Informática este año he procedido a realizar un trabajo por cada uno de los grados cursados y este TFG busca ser la primera parte de una investigación un poco más extensa. La segunda parte de dicha investigación se corresponde con el TFG realizado en colaboración con mi compañero Luis Aguirre para la Facultad de Informática cuyo título es \textit{Computación Cuántica: Pruebas de Mutación}. En el hemos usado las nociones de este trabajo y el realizado por mi compañero para el Grado de Matemáticas para indagar en las aplicaciones del \textit{testing} sobre la computación cuántica, concretamente de las posibles aplicaciones del \textit{mutation testing} en esta rama de la computación. Ambos trabajos, con muchos nexos en común, se complementan y se entienden mejor como un todo y es así como me gustaría mostrarlos al lector. Sin embargo, debido a la evaluación individual que debe ser realizada sobre cada uno, sólo puedo recomendar la lectura del TFG antes mencionado si el lector sintiera mayor interés por descubrir nuevos conocimientos acerca de este tema.

Por otro lado, el TFG tampoco es una guía de iniciación a \textit{Qiskit}. Hablaremos sobre la estructura principal de sus librerías, introduciremos brevemente su sintaxis  y presentaremos algunos ejemplos, pero evitaremos formalidades propias de la sintaxis, ejecución y las demás características que definen un lenguaje de programación. La principal razón por la que no hacemos hincapié en estos aspectos es porque más que un lenguaje debe considerarse como un \textit{framework} de computación cuántica para trabajar en \textit{Python} y por tanto deberíamos hablar de las propiedades de este lenguaje en lugar de las propias de \textit{Qiskit}. Además, presentamos en el apéndice~\ref{ap:ap1} los experimentos cuánticos de codificación superdensa y teleportación y una clase  para experimentar con el algoritmo de \textit{Deutsch-Jozsa}. Dicha clase incluye algunas funcionalidades básicas para facilitar al usuario la creación de un circuito del número de qubits deseado computable en un simulador o, incluso, en un ordenador cuántico real.

Tras una breve introducción histórica, que ocupa el primer capítulo de esta memoria, los capítulos segundo y tercero se centran en sentar una cierta, aunque sólida, base matemática para poder asentar los conceptos de la mecánica cuántica en los que apoyamos toda la teoría de la computación cuántica. Dicho conocimiento matemático es, esencialmente, nociones del álgebra lineal de sobra conocidas. El grueso de la investigación presentada en esta memoria se encuentra en el capítulo cuatro, donde introduciremos brevemente una versión ligera de los postulados cuánticos y toda la teoría de la información cuántica.
%
Así, establecemos una pirámide de conocimientos cuya base son las matemáticas, su cúspide la computación cuántica y en un nivel intermedio situamos a la física. Por esto quiero destacar que, como la propia computación clásica en general, esta área de conocimiento es muy apropiada para tratar en un trabajo por un alumno de este doble grado y, pese al gran tiempo dedicado, ambos TFGs realizados han resultado muy satisfactorios a nivel personal.

\chapter*{Objetivos y plan de trabajo}
\addcontentsline{toc}{chapter}{Objetivos y plan de trabajo}

Como se menciona en el prefacio, este no es el único TFG realizado por mí sobre computación cuántica. Esto hace que resulte difícil diferenciar objetivos y, especialmente, planificación para uno y otro trabajo.
%
Podemos establecer que los objetivos principales de este TFG consisten en tener una base formada sobre las normas que rigen la computación cuántica y cimentarlas con los conocimientos matemáticos y físicos que sustentan dichas normas. La adquisición de estos conocimientos sirve para catapultarse en el mundo del \textit{testing} aplicado a la computación cuántica, tema tratado en el otro TFG. Si la computación cuántica es un área de investigación prácticamente nueva tanto a nivel \textit{hardware} como \textit{software}, más aún lo es el \textit{testing} aplicado a la misma en la que todavía se están identificando cómo trasladar las técnicas de la computación clásica a la cuántica y se están diseñando otras nuevas.

Otro objetivo evidente es el manejo, al menos a nivel básico, del lenguaje \textit{Qiskit} más allá de lo reflejado en este TFG. Conocer los métodos y técnicas de programación más utilizados en un ámbito en el que apenas existen algoritmos ni código no es una tarea sencilla. Sin embargo, es primordial tener una mínima idea de estos entresijos para poder realizar \textit{testing} sobre el lenguaje.

Por último, aunque está lejos de ser un eje central de esta investigación, podemos considerar objetivo el familiarizarse con la situación actual de los ordenadores cuánticos. Su disponibilidad para un usuario novel, simuladores, el número de qubits que presentan actualmente o la tasa de error presentes en sus cómputos son temas mencionados en estas páginas.

En cuanto al plan de trabajo, consta de una primera etapa de investigación que requiere bastante tiempo. Hay que tener en cuenta que la mayoría de conocimientos son nuevos y nada sencillos. En lo referente a este trabajo, la investigación realizada incluía álgebra lineal, mecánica cuántica, computación cuántica y la documentación sobre el lenguaje \textit{Qiskit}. Me parece importante recalcarlo porque no siempre es fácil reflejar en una memoria todo el conocimiento adquirido, que en este caso ha sido mucho mayor que el presentado en dicha memoria.
%
La segunda etapa no era propia de este TFG. Incluía más investigación sobre \textit{testing} y el desarrollo de un sistema para aplicar pruebas a código cuántico que también requirió de bastante tiempo. A continuación se realizaron las memorias de ambos TFG, a menudo intercalando esfuerzos en uno y otro. Por último, se iteran varias revisiones de las memorias hasta su versión final.

Para concluir, quiero destacar mis fuentes principales para la adquisición de los conocimientos plasmados en esta memoria. El comienzo fue la lectura de \textit{An Introduction to Quantum Computing for Non-Physicists} [\cite{rieffel2000introduction}] que evita la mayoría de conocimientos matemáticos y se adentra rápidamente en definir la teoría de la computación cuántica. Un libro mucho más completo es \textit{Quantum Computation and Quantum Information} [\cite{nielsen2001quantum}]. Aunque algunos conceptos propios de la mecánica cuántica no se presentan con suficiente profundidad, contiene todo el conocimiento necesario para tener una base más que sólida sobre la computación cuántica, además de presentar contenido más avanzado. Aunque podríamos destacar otros incluidos en la bibliografía al final de este TFG, estos dos constituyen el grueso en cuanto a computación cuántica se refiere.