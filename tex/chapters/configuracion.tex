\documentclass[11pt, oneside]{book}


\usepackage[utf8]{inputenc}
\usepackage[spanish]{babel}

%--------Codigos para la caligrafia, tipos de letras%---------------
\usepackage[T1]{fontenc}
%\usepackage[bitstream-charter]{mathdesign} 
%\usepackage{textcomp} %Paquete para algunos caracteres especiales

\usepackage{afterpage}
\usepackage{mathtools}

\usepackage{amssymb}
\usepackage[usenames]{color}
\usepackage{enumerate}
\usepackage{stackrel}
\usepackage{amsthm}
\usepackage{centernot}
\usepackage[a4paper, total={6in, 8in}]{geometry}
\usepackage{pgfplots}
\usepackage[toc,page]{appendix}
\usepackage{graphicx} % graficos



\newtheorem{teor}{Teorema} % Estilo de texto cursivo

\theoremstyle{definition} % Cambio del estilo de los teoremas a normal
\newtheorem{proposicion}{\textbf{Proposici\'on}}
\newtheorem{proposicioni}{\textbf{Proposici\'on \textit{(Teorema de la función inversa)}}}
\newtheorem{defi}{\textbf{Definición}}
\newtheorem{corolario}{Corolario}
\newtheorem{figura}{Figura}
\newtheorem*{lema}{Lema}
\newtheorem*{ejem}{\underline{Ejemplo}}
\newtheorem*{nota}{Nota}
\newtheorem*{observacion}{Observación}
%\renewenvironment{proof}{{\bfseries Demostración}}{\begin{flushright}\qedsymbol\end{flushright}} %Demostración en negrita

\newcommand{\talque}{\mathrm{\ _\shortparallel\ }}
\newcommand{\R}{\mathbb{R}}
\newcommand{\e}{\mathrm{e}}
\newcommand{\N}{\mathbb{N}}
\newcommand{\Q}{\mathbb{Q}}
\newcommand{\C}{\mathbb{C}}
\newcommand{\Cgot}{\mathcal{C}}
\newcommand{\dom}{\mathrm{Dom\ }}
\newcommand{\upint}[2]{\overline{\displaystyle\int_{#2}^{#1}}}
\newcommand{\lowint}[2]{\underline{\displaystyle\int_{#2}^{#1}}}
\newcommand{\integral}[2]{\displaystyle\int_{#2}^{#1}}
\newcommand{\gradiente}{\bigtriangledown}
\newcommand{\nimplies}{\centernot\implies}
\newcommand{\nimpliedby}{\centernot\impliedby}
\newcommand{\nlongrightarrow}{\centernot\longrightarrow}
\newcommand{\nsubset}{\centernot\subset}
\newcommand{\oversim}[1]{\stackbin{_\sim}{#1}}
\newcommand{\sucesion}[2]{\left\{#1_{#2}\right\}^\infty_{#2 = 1}}
\newcommand{\sucesionelement}[2]{\left\{#1\right\}^\infty_{#2 = 1}}
\newcommand{\doubleright}[2]{ \left. \begin{array}{ll}	#1 \\	#2 \\ \end{array} 	\right\} }
\newcommand{\doubleleft}[2]{ \left\{\begin{array}{ll}	#1 \\	#2 \\  \end{array}	\right. }
\newcommand{\doubleleftright}[2]{ \left\{\begin{array}{ll} #1 \\ #2 \\ \end{array} \right\}}
\newcommand{\double}[2]{ \left. \begin{array}{ll}	#1 \\	#2 \\	 \end{array} \right. }	
\newcommand{\tripleright}[3]{ \left. \begin{array}{ll}	#1 \\#2 \\#3\\	\end{array} \right\} }
\newcommand{\tripleleft}[3]{ \left\{ \begin{array}{ll} #1 \\#2 \\#3\\ \end{array} \right. }
\newcommand{\triple}[3]{ \left. \begin{array}{ll}	#1 \\#2 \\#3\\	\end{array} 	\right. }
\newcommand{\ximplies}[2]{\stackbin[#2]{#1}\implies}
\newcommand{\xiff}[2]{\stackbin[#2]{#1}\iff}
\newcommand{\ximpliedby}[2]{\stackbin[#2]{#1}\impliedby}
\newcommand{\dotproduct}[2]{<#1,#2>}
\newcommand{\norm}[1]{\left|\left|#1\right|\right|}
\newcommand{\function}[3]{#1\colon #2\longrightarrow #3}
\newcommand{\xfunction}[4]{\stackbin[#4]{}{#1\colon #2\longrightarrow #3}}
\newcommand{\equals}[2]{\underset{#2}{\underset{\shortparallel}{#1}}}
\newcommand{\limite}[2]{\stackbin[#2]{\ }\lim\ #1}
\newcommand{\limited}{\stackbin{n\rightarrow\infty}\longrightarrow}
\newcommand{\xtiende}{x\rightarrow x_0}
\newcommand{\ntiende}{n\rightarrow\infty}
\newcommand{\ttiende}{t\rightarrow 0}
\newcommand{\htiende}{h\rightarrow 0}
\newcommand{\ktiende}{k\rightarrow 0}
\newcommand{\y}{\mathrm{\ y\ }}
\newcommand{\cte}{\mathrm{cte}}
\newcommand{\rg}{\mathrm{rg}}
\newcommand{\id}{\mathrm{id}}
\newcommand{\diam}{\mathrm{diam}}
\newcommand{\ais}{\mathrm{Ais}}
\newcommand{\fr}{\mathrm{Fr}}
\newcommand{\vacio}{\emptyset}
\newcommand{\deter}[1]{\left| \begin{matrix} #1	\end{matrix}\right|}
\newcommand\blankpage{%
    \null
    \thispagestyle{empty}%
    \addtocounter{page}{-1}%
    \newpage}




\title{C\'alculo Integral}


\author{Javier Pellejero Ortega}