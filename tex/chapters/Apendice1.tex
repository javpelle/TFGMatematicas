\chapter{Espacios normados y con producto interno}

Para definir con precisión qué es un espacio de Hilbert, necesitamos definir una serie de conceptos previos básicos. Dichos conceptos son estudiados en los primeros cursos del grado de Matemáticas.

Pese a que son definiciones y desarrollos ampliamente conocidos, son expuestos a continuación los estrictamente necesarios por si el lector precisara recordarlos. El material presentado se extrae fundamentalmente de Lokenath Debnath, Piotr Mikusinski y col. \textit{Introduction to Hilbert spaces with applications} y de las notas propias cosechadas en diferentes asignaturas.

\section{Espacios vectoriales}

Habitualmente se estudian tanto espacios vectoriales reales como complejos. Denotamos $\R$ al cuerpo de los reales y $\C$ al de los complejos y nos referiremos a cada uno de sus elementos como \textbf{escalar}. En caso de que no queremos hacer distinción del cuerpo con el que trabajemos lo denotaremos por $\K:=\R$ o $\C$.

\begin{definition} Un \textbf{espacio vectorial} es un conjunto no vacío $E$ que consta de las operaciones:
\begin{itemize}
\item \textbf{Suma}. $(x,y)\mapsto x+y$ de $E\times E$ en $E$.
\item \textbf{Multiplicación por escalar}. $(\lambda,y)\mapsto \lambda x$ de $\K\times E$ en $E$.
\end{itemize}
que verifican para todo $x,y,z\in E$ y para todo $\alpha,\beta\in\K$ que:
\begin{enumerate}[label=\alph*)]
\item $x + y = y + x$.
\item $(x + y) + z = x + (y + z)$.
\item Para todo $x,y\in E$ existe $z\in E$ tal que $x+z=y$.
\item $\alpha(\beta x)=(\alpha\beta)x$.
\item $(\alpha + \beta)x=\alpha x +\beta x$.
\item $\alpha (x + y) =\alpha x +\alpha y$.
\item $1x = x$.
\end{enumerate}
\end{definition}

De este definición se obtienen algunos resultados interesantes que no comentaremos aquí. Nos limitaremos a ejemplificar qué ocurre en el caso de los complejos que es el cuerpo con el que trataremos.

\begin{example} $\K=\R$ o $\C$ son los espacios vectoriales más simples con las operaciones suma y multiplicación habituales. Además podemos definir $\R^n$ y $\C^n$ como:
\[\R^n:=\{(x_1,x_2,...,x_n): x_1,x_2,...,x_n\in\K\}\]
\[\C^n:=\{(z_1,z_2,...,z_n): z_1,z_2,...,z_n\in\K\}\]
En particular, $\C^{2^n}$ es un espacio vectorial.
\end{example}

\section{Espacios normados}

\begin{definition} Sea $E$ un espacio vectorial, denominamos a la función $\function{\norm\cdot}{E}{\R}$ tal que $x\mapsto\norm{x}$ como \textbf{norma} si verifica:
\begin{enumerate}[label=\alph*)]
\item Si $\norm{x}=0$ entonces $x=0$.
\item $\norm{\lambda x}=|\lambda|\norm{x}$ para todo $\lambda\in\K$ y para todo $x\in E$.
\item $\norm{x+y}\leq \norm{x} + \norm{y}$ para todo $x,y\in E$, que denominamos \textbf{desigualdad triangular}.
\end{enumerate}

Denotamos como \textbf{espacio normado} a un espacio vectorial dotado de norma.
\end{definition}

\begin{definition} Sea una sucesión de elementos $\{x_n\}$ de un espacio normado, decimos que es una \textbf{sucesión de Cauchy} si verifica que para todo $\varepsilon>0$ existe un $M>0$ tal que $\norm{x_m-x_n}<\varepsilon$ para todos $m,n>M$.
\end{definition}

\begin{definition} Decimos que es un espacio normado es \textbf{completo} si toda sucesión de Cauchy en $E$ converge a un elemento de $E$. En dicho caso, denominamos a tal espacio normado \textbf{espacio de \textit{Banach}}.
\end{definition}

\begin{example} $\K=\R$ o $\C$ son los espacios vectoriales más simples con las operaciones suma y multiplicación habituales. Además podemos definir $\R^n$ y $\C^n$ como:
\[\R^n:=\{(x_1,x_2,...,x_n): x_1,x_2,...,x_n\in\K\}\]
\[\C^n:=\{(z_1,z_2,...,z_n): z_1,z_2,...,z_n\in\K\}\]
En particular, $\C^{2^n}$ es un espacio vectorial.
\end{example}

\section{Espacios con producto interno}

\begin{definition} Sea $E$ un espacio vectorial complejo, definimos la aplicación $\function{\dotproduct{\cdot}{\cdot}}{E\times E}{\C}$ como \textbf{producto interior} o \textbf{producto escalar} de $E$ si verifica para todo $x,y,z\in E$ y para todo $\alpha,\beta\in\C$:
\begin{enumerate}[label=\alph*)]
\item $\dotproduct{x}{y}=\overline{\dotproduct{y}{x}}$.
\item $\dotproduct{\alpha x + \beta y}{z}= \alpha \dotproduct{x}{y} + \beta \dotproduct{y}{z}$.
\item $\dotproduct{x}{x}\geq 0$.
\item $\dotproduct{x}{x}= 0$ implica que $x=0$.
\end{enumerate}

Decimos que un espacio vectorial (no necesariamente complejo) provisto de producto interno es un \textbf{espacio con producto interno} o \textbf{espacio prehilbert}.
\end{definition}

\begin{observation} Hemos definido el producto interno sobre un espacio vectorial complejo; Sin embargo, no hay problema si lo sustituimos por uno real en la definición. Basta definir el producto como $\function{\dotproduct{\cdot}{\cdot}}{E\times E}{\R}$ y ver que en la primera condición se verifica
\[\dotproduct{x}{y}=\overline{\dotproduct{y}{x}}=\dotproduct{y}{x}.\]

El resto de condiciones están bien definidas en el caso de $\R$.
\end{observation}

\begin{proposition} Todo espacio con producto interno es también un espacio normado con la norma $\norm{x}:=\sqrt{\dotproduct{x}{x}}$.
\begin{proof} Tenemos que probar las tres condiciones vistas en la \textbf{Definición A.3}.
\begin{enumerate}[label=\alph*)]
\item $0 = \norm{x} = \sqrt{\dotproduct{x}{x}}$ implica que $x=0$.
\item $\norm{\lambda x} = \sqrt{\dotproduct{\lambda x}{\lambda x}} = \sqrt{\lambda\overline{\lambda}\dotproduct{x}{x}}=|\lambda|\norm{x}$.
\item Para la tercera condición necesitamos hacer uso de la \textbf{desigualdad de Schwarz} que presentamos a continuación. No incluiremos su demostración pese a no tener dificultad.

\begin{thm} \textbf{Desigualdad de Schwarz}. Sean $x$ e $y$ cualquiera de un espacio dotado de producto interno, tenemos:
\begin{equation}
|\dotproduct{x}{y}|\leq\norm{x}\norm{y}.
\end{equation}
\end{thm}
Una vez introducida la desigualdad, tenemos:
\[\norm{x+y}^2 = \dotproduct{x}{x} + \dotproduct{x}{y} + \overline{\dotproduct{x}{y}} + \dotproduct{y}{y} = \dotproduct{x}{x} + 2\mathrm{Re}\dotproduct{x}{y} + \dotproduct{y}{y} \leq\]
\[\leq \dotproduct{x}{x} + 2|\dotproduct{x}{y}| + \dotproduct{y}{y} = \norm{x}^2+ 2|\dotproduct{x}{y}|+\norm{y}^2\leq\]
\[\overset{\mathrm{(A\centerdot1)}}{\leq}\norm{x}^2+ 2\norm{x}\norm{y}+\norm{y}^2 = \left(\norm{x}+\norm{y}\right)^2\]
Lo que concluye la demostración.
\end{enumerate}
\end{proof}
\end{proposition}