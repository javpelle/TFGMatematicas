\chapter{Introducción histórica}

Comencemos dando unas pinceladas sobre la relativamente corta historia de la computación cuántica.

\section{Historia de la computación cuántica}
La computación cuántica se sustenta en las leyes de la mecánica cuántica que empezaron a desarrollarse a principios del siglo XX. Sin embargo, no fue hasta 1982 cuando se planteó el uso de sus propiedades en la computación. El premio \textit{Nobel} Richard Feynman estudiaba el modelado para la simulación de efectos cuánticos en un ordenador clásico y observó que algunos de dichos efectos no podían simularse eficientemente. Se planteó entonces si el uso de las propiedades de la mecánica cuántica en el mundo computacional podrían suponer un significativo aumento del poder de cálculo de los ordenadores clásicos [\cite{feynman1982modeling}].

Estas suposiciones de Feynman generaron un pequeño interés meramente teórico; llevarlo a la práctica suponía construir máquinas que aprovecharan los efectos cuánticos, lo cual se antojaba difícil y además, dentro del marco teórico, tampoco se sabía demasiado bien cómo formular algoritmos cuánticos que mejoraran los clásicos.

Fue en 1994 cuando Peter Shor publicó un algoritmo cuántico que factorizaba enteros en tiempo polinómico [\cite{shor1994algorithms}]. Esto supuso una pequeña revolución, los algoritmos criptográficos más importantes se basan en la dureza del problema de factorizar enteros de gran tamaño que tiene una complejidad exponencial.

A raíz del logro de Shor, multitud de científicos se interesaron por el estudio de la computación cuántica y comenzó así una carrera que perdura hasta nuestros días para construir ordenadores cuánticos capaz de ejecutar este y otros algoritmos que empezaron a surgir.

Previamente al gran descubrimiento de Shor, David Deutsch y Richard Jozsa ya habían llamado a la puerta de los algoritmos cuánticos, en 1992 concretamente [\cite{deutsch1992rapid}]. Presentaron un algoritmo cuántico relativamente sencillo para un problema concreto: sea una función $\function{f}{\{0,1\}^n}{\{0,1\}}$ de la que sólo sabemos que, o bien es constante o bien es balanceada (devuelve 0 para la mitad de las entradas y 1 para la otra mitad), queremos determinar de qué tipo es.

Pese a lo inocente del problema, si queremos resolverlo de manera clásica, nos encontramos con un algoritmo que debe verificar en caso peor $2^{n-1}+1$ entradas; Es decir, tiene complejidad exponencial. El de Deutsch y Jozsa tenía coste constante, tan sólo hace falta una evaluación. Esta mejora nos da una idea del poder computacional que podría ofrecer un ordenador cuántico.

A lo largo de estos años han surgido otros potentes algoritmos y se han logrado hazañas en el campo práctico con la consecución de los primeros ordenadores cuánticos. Además, el inexorable avance de la computación clásica también nos ha permitido diseñar simuladores en los que podemos probar nuevos programas, aunque obviamente, sin la mejora computacional que ofrecería ejecutarlos sobre un computador cuántico.

\section{La computación cuántica en la actualidad}
Aunque podríamos hablar de los numerosos avances llevados a cabo a lo largo del siglo XXI, el pasado año 2019 fue tal vez en el que más expectación se ha levantado en cuanto avances. Por parte de IBM se reveló por un lado el diseño del primer ordenador cuántico comercial, el \textit{IBM Q System One} que consta de un total de 20 qubits. Por otro lado, anunció la puesta en funcionamiento de un ordenador cuántico de 53 qubits, el mayor hasta la fecha, para octubre de 2019.

Pese a estos notables avances, la gran noticia del año fue una que muchos esperaban pues Google llevaba un tiempo anunciándola: la llegada de la supremacía cuántica. Este es un término para referirse a la capacidad real de un ordenador cuántico en realizar una tarea en la que uno clásico no podría resolver en un tiempo prudencial.

En octubre de 2019 se publicó un artículo [\cite{arute2019quantum}] en que se afirmaba tal noticia. En concreto, se decía que se había logrado ejecutar una tarea en unos 200 segundos en un ordenador cuántico mientras que, en uno clásico, se hubiera ejecutado en un tiempo estimado de 10.000 años.

Esta información no estuvo exenta de polémica, al poco tiempo de su publicación dejó de estar disponible y otras grandes compañías en la carrera por la supremacía cuántica, como la propia IBM, negaron tal hazaña argumentando que la estimación de 10.000 años referida por Google estaba muy lejos de la realidad.

Cabe destacar que, pese a los esfuerzos y logros cosechados en los últimos años en este campo, aún queda mucho camino por recorrer. Algunas voces del mundo científico, como el matemático israelí Gil Kalai, incluso afirman y argumentan que no será posible lograr nunca un ordenador cuántico que en la práctica sea útil y que todos los esfuerzos actuales serán en vano [\cite{kalai2011quantum}].