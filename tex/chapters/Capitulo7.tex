\chapter{Conclusiones}
\label{chapter7}

Para concluir esta memoria, vamos a aportar algunas conclusiones y comentarios acerca de los temas tratados.
%
Empezando por lo más reciente, en la sección~\ref{sec:sec61} mostramos  los resultados obtenidos por un simulador y por un ordenador cuántico para dos aplicaciones del algoritmo Deutsch-Jozsa y su tasa de error de alrededor del 50\% cuando es ejecutado en el segundo. En el caso de un algoritmo determinista como este, el problema se soluciona ejecutando varias veces el mismo circuito y tomando el valor con mayor número de repeticiones observando que el resto de resultados se obtienen con una frecuencia mucho menor y con cierta uniformidad. Si la tasa de error se mantiene constante en torno al 50\%, independientemente del tamaño de la entrada, dicho error no afecta al orden de complejidad del algoritmo, pues el número de ejecuciones a realizar sería una constante. Lamentablemente, esto no es así. En nuestro caso, nuestro ordenador constaba de 5 qubits así que ya hacíamos uso de toda la entrada disponible y el desarrollo de ordenadores con un número de qubits mucho más elevado está siendo un proceso lento debido, precisamente, a la creciente inestabilidad surgida con el aumento de los qubits.

Este problema se acrecienta si tenemos en cuenta que muchos de los algoritmos cuánticos son probabilistas, como los de \textit{Grover} o \textit{Shor}. En estos casos puede ocurrir que  resultados no buscados no tengan amplitud cero,  que existan múltiples resultados satisfactorios con amplitudes ``altas'' pero ninguno con amplitud uno o, en el peor de los casos, que se den ambas situaciones simultáneamente. El éxito de estos algoritmos depende de la realización de múltiples ejecuciones para descartar resultados erróneos. Esto, sumado al error mencionado anteriormente, puede aumentar considerablemente el número de ejecuciones necesarias para obtener resultados fiables, poniendo en entredicho la eficiencia del algoritmo.

Por otro lado, no sólo hay que destacar el gran trabajo que queda por delante en términos de \textit{hardware} cuántico, sino también en cuanto a algoritmia. Ya mencionamos cuando hablábamos de \textit{Qiskit Aqua}, la librería de \textit{Qiskit} dirigida a algoritmos cuánticos, que su contenido era reducido. Si bien es cierto que la consecución de algoritmos cuánticos que mejoren a los clásicos no es una tarea sencilla, debido a la complejidad y falta de intuición que manejan estos conocimientos, conocemos multitud de algoritmos clásicos realmente complejos. Personalmente creo que, además de la dificultad presentada, existen otros impedimentos como la incertidumbre de si algún día tendremos máquinas capaces de ejecutar dichos algoritmos eficientemente o el cambio radical en el paradigma de programación.

Evidentemente no todo es negativo. En estas páginas se deja entrever el poder de cómputo que pueden llegar a alcanzar estas máquinas. Si en los próximos años se lograra un ordenador cuántico con un número de qubits suficientemente alto, podríamos contemplar una revolución en el mundo de las comunicaciones. Ya mencionamos que el algoritmo de Shor pondría en apuros la seguridad de cualquier comunicación en línea cifrada por el sistema criptográfico RSA o similares, por lo que estaríamos obligados a rediseñar la seguridad de todas ellas. Desde luego no sería la única utilidad de tal hazaña y desconocemos los recursos y algoritmos cuánticos que quedan por descubrir antes de que dicha gesta ocurra.