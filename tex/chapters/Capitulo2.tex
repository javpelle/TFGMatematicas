\chapter{Introducción a la computación cuántica}

Antes de adentrarnos en el marco teórico de la computación cuántica es conveniente tratar con las bases matemáticas que lo sustentan. Evitaremos en la mayor medida posible las cuestiones propias de la física de las que se nutre la mecánica cuántica y por ende la computación cuántica.

Toda esta física esconde muchas matemáticas que, por tanto, no serán presentadas en este capítulo. Pese a ello, es idóneo ver los cimientos de dicho marco teórico que podemos resumir en las propiedades y operadores que presentan los números complejos como espacio de \textit{Hilbert}.

\section{Espacios de Hilbert}
\begin{definition} Definimos un \textbf{espacio de \textit{Hilbert}} como un espacio dotado de producto interior cuyo espacio normado que define con la norma $\norm{x}:=\sqrt{\dotproduct{x}{x}}$ es completo (ver \textbf{Apéndice A}).
\end{definition}

