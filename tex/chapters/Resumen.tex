\chapter*{}
\section*{Resumen}
\addcontentsline{toc}{chapter}{Resumen}

Esta memoria pretende reflejar una introducción a la computación cuántica, haciendo énfasis en la teoría matemática que la sustenta. Construiremos toda esta teoría empezando por los elementos más básicos del álgebra lineal, continuando por aplicaciones lineales y productos tensoriales en espacios de \textit{Hilbert} y dando un breve vistazo a los postulados cuánticos. Este camino es lo más directo posible, obviando conceptos del álgebra y la física no necesarios aunque interesantes. Tras la explicación de dichos conocimientos, hablaremos de la teoría de la computación cuántica, sus elementos, operadores y principales propiedades para más tarde introducirnos en el lenguaje de computación cuántica \textit{Qiskit} y concluyendo con algunos algoritmos e implementaciones.

\section*{Abstract}
\addcontentsline{toc}{chapter}{Abstract}

This report aims to reflect an introduction to quantum computing, emphasizing the mathematical theory behind it. We will build this entire theory starting with the most basic elements of linear algebra, continuing with linear applications and tensor products in \textit{Hilbert} spaces and giving a brief look at quantum postulates. This path is as direct as possible, avoiding unnecessary but interesting concepts of algebra and physics. After this, quantum computing theory will be discussed, its elements, operators and main properties. We will introduce the quantum computing language Qiskit and conclude with some algorithms and implementations.